\documentclass[11pt]{article}

\usepackage{report}

\usepackage[utf8]{inputenc}
\usepackage[T1]{fontenc}
\usepackage[colorlinks=true, linkcolor=black, citecolor=blue, urlcolor=blue]{hyperref}
\usepackage{url}
\usepackage{booktabs}
\usepackage{amsfonts}
\usepackage{amsmath}
\usepackage{nicefrac}
\usepackage{microtype}
\usepackage{graphicx}
\usepackage{natbib}
\usepackage{doi}
\usepackage{minted}
\setcitestyle{aysep={,}}



\title{Predicting Food Groups \\ \vspace{0.5em} \large from nutritional information}

\author{
    David Sha\\
\AND
    Victoria Lyngaae\\
\AND
    Mason Sebek\\
\AND
    William Spongberg\\
\AND
\AND
	COMP20008: Elements of Data Processing\\
\AND
	The University of Melbourne\\
}

\date{May 2023}

\renewcommand{\headeright}{Predicting Food Groups from Nutritional Information}
\renewcommand{\undertitle}{Report}
\renewcommand{\shorttitle}{}

\begin{document}
\maketitle

\newpage
\tableofcontents
\thispagestyle{empty}

\newpage
\setcounter{page}{1}
\section{Introduction}

In this report, we will be analysing a food dataset from \cite{FoodStandardsAustraliaNewZealand}. Using the \(k\)-nearest neighbours algorithm, we build a model that predicts the food groups of a food item based on its nutritional information and explore how useful this model is to the food industry.

\subsection{Research Question}
% 1. What is the research question?
Can we predict the food group of a food item based solely on its nutritional information?

\subsection{Target Audience}
% 2. Who are the target audience.
The outcome of this project may be interesting to supermarket chains. If the model is sufficiently accurate, it could be used to detect food items that may not be labelled correctly or are missing labels. Better classification of food items can improve a supermarket's inventory management and/or product placement and ultimately improve the shopping experience of customers.

\subsection{Dataset}
% 3. The dataset you have chosen.
We make use of the dataset from \cite{FoodStandardsAustraliaNewZealand} containing 1616 food items and their nutritional information. This dataset contains all the information we need to answer our research question as each food item has a \verb|Classification| (food group) and respective nutritional information (such as energy, protein, etc.). To give us more context to the \verb|Classification| column, we also use a complementary dataset from \cite{FoodClassification} which maps numerical classifications to their respective food groups which become very useful when preprocessing the data and analysing the results.

\section{Methodology}

\emph{What wrangling and analysis methods (including at least one supervised learning method) have you applied? Why have you chosen these methods over other alternatives? How do you perform the experiment?}

\subsection{Data Wrangling}

\subsection{Data Analysis}

\subsection{Supervised Learning}

\section{Results}
\emph{What are the key results your research has obtained?}

\section{Discussion}
\emph{Why are your results significant and valuable?}

\section{Conclusion}
\emph{What are the limitations of your results and how can the project be improved for future?}

\newpage
\bibliographystyle{plainnat}
\bibliography{references}


\end{document}
