\documentclass[11pt]{article}

\usepackage{report}

\usepackage[utf8]{inputenc}
\usepackage[T1]{fontenc}
\usepackage[colorlinks=true, linkcolor=black, citecolor=blue, urlcolor=blue]{hyperref}
\usepackage{url}
\usepackage{booktabs}
\usepackage{amsfonts}
\usepackage{amsmath}
\usepackage{nicefrac}
\usepackage{microtype}
\usepackage{graphicx}
\usepackage{natbib}
\usepackage{doi}
\usepackage{minted}
\setcitestyle{aysep={,}}



\title{TODO: Come up with a title}

\author{
    David Sha\\
\AND
    Victoria Lyngaae\\
\AND
    Mason Sebek\\
\AND
    William Spongberg\\
\AND
\AND
\AND
	COMP20008: Elements of Data Processing\\
\AND
	The University of Melbourne\\
}

\date{May 2023}

\renewcommand{\headeright}{TODO: Come up with a title}
\renewcommand{\undertitle}{Report}
\renewcommand{\shorttitle}{}

\begin{document}
\maketitle

\newpage
\tableofcontents
\thispagestyle{empty}

\newpage
\setcounter{page}{1}
\section{Introduction}

In this report, we will be analysing a food dataset from \cite{FoodStandardsAustraliaNewZealand}.

\subsection{Research Question}
% 1. What is the research question?

\subsection{Target Audience}
% 2. Who are the target audience.

\subsection{Dataset}
% 3. The dataset you have chosen.

\section{Methodology}

\emph{What wrangling and analysis methods (including at least one supervised learning method) have you applied? Why have you chosen these methods over other alternatives? How do you perform the experiment?}

\subsection{Data Wrangling}

\subsection{Data Analysis}

\subsection{Supervised Learning}

\section{Results}
\emph{What are the key results your research has obtained?}

\section{Discussion}
\emph{Why are your results significant and valuable?}

\section{Conclusion}
\emph{What are the limitations of your results and how can the project be improved for future?}

\newpage
\bibliographystyle{plainnat}
\bibliography{references}


\end{document}
